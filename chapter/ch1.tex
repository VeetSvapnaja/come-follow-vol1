\chapter{Dan Firman itu Telah Menjadi Manusia}

\bahasa
21 Oktober 1975 pagi di Aula Buddha

\english
21 October 1975 am in Buddha Hall

\bahasa
Yohanes 1

\english
JOHN 1

\bahasa
1 PADA MULANYA ADALAH FIRMAN, DAN FIRMAN ITU ADA BERSAMA TUHAN, DAN FIRMAN ITU ADALAH TUHAN

\english
1 In The Beginning Was The Word, And The Word Was With God, And The Word Was God

\bahasa
3 SEGALA SESUATU DIJADIKAN OLEH DIA DAN TANPA DIA TIDAK ADA SUATUPUN YANG TELAH JADI DARI SEGALA YANG TELAH DIJADIKAN

\english
3 ALL THINGS WERE MADE BY HIM AND WITHOUT HIM WAS NOT ANYTHING MADE THAT WAS MADE.

\bahasa
4 DI DALAM DIA ADALAH HIDUP; DAN HIDUP ADALAH TERANG MANUSIA.

\english
4 IN HIM WAS LIFE; AND LIFE WAS THE LIGHT OF MEN.

\bahasa
5 DAN TERANG ITU BERCAHAYA DALAM KEGELAPAN; DAN KEGELAPAN TIDAK DAPAT MEMAHAMINYA.

\english
5 AND THE LIGHT SHINETH IN DARKNESS; AND THE DARKNESS COMPREHENDED IT NOT.

\bahasa
6 ADA SEORANG YANG DIUTUS OLEH TUHAN, YANG BERNAMA YOHANES.

\english
6 THERE WAS A MAN SENT FROM GOD, WHOSE NAME WAS JOHN.

\bahasa
7 IA DATANG UNTUK MENJADI SAKSI, UNTUK MEMBERI KESAKSIAN TENTANG TERANG ITU, SEHINGGA SEMUA MANUSIA MELALUI DIA MENJADI PERCAYA.

\english
8 THE SAME CAME FOR A WITNESS, TO BEAR WITNESS OF THE LIGHT, THAT ALL MEN THROUGH HIM MIGHT BELIEVE.

\bahasa
8 IA BUKAN TERANG ITU, TETAPI DIUTUS UNTUK MEMBERI KESAKSIAN TENTANG TERANG ITU.

\english
8 HE WAS NOT THAT LIGHT, BUT WAS SENT TO BEAR WITNESS OF THAT LIGHT.

\bahasa
11 IA DATANG KEPADA MILIKNYA SENDIRI, DAN MILIKNYA ITU SENDIRI TIDAK MENERIMANYA.

\english
11 HE CAME UNTO HIS OWN,AND HIS OWN RECEIVED HIM NOT.

\bahasa
12 TETAPI SEBERAPA BANYAK ORANG YANG MENERIMA DIA, KEPADA MEREKA ITULAH DIBERINYA KUASA UNTUK MENJADI ANAK-ANAK ALLAH, BAHKAN BAGI MEREKA YANG PERCAYA KEPADA NAMANYA.

\english
12 BUT AS MANY AS RECEIVED HIM, TO THEM GAVE HE POWER TO BECOME THE SONS OF GOD, EVEN TO THEM THAT BELIEVE ON HIS NAME:

\bahasa
14 DAN FIRMAN ITU TELAH MENJADI MANUSIA DAN TINGGAL DI TENGAH-TENGAH KITA, (DAN KITA MELIHAT KEMULIAANNYA, KEMULIAAN DARI SATU-SATUNYA YANG MENJADI ANAK BAPA), PENUH DENGAN ANUGERAH DAN KEBENARAN.

\english
14 AND THE WORD WAS MADE FLESH, AND DWELT AMONG US, (AND WE BEHELD HIS GLORY, THE GLORY AS OF THE ONLY BEGOTTEN OF THE FATHER), FULL OF GRACE AND TRUTH.

\bahasa
16 DAN DARI KEPENUHANNYA KITA SEMUA MENERIMA, ANUGERAH DEMI ANUGERAH

\english
16 AND OF HIS FULLNESS HAVE ALL WE RECEIVED, AND GRACE FOR GRACE.

\bahasa
17 KARENA HUKUM DIBERIKAN OLEH MUSA, TETAPI KASIH KARUNIA DAN KEBENARAN DATANG OLEH YESUS KRISTUS.

\english
17 FOR THE LAW WAS GIVEN BY MOSES, BUT GRACE AND TRUTH CAME BY JESUS CHRIST.

\bahasa
AKU AKAN BERBICARA TENTANG KRISTUS, tapi tidak tentang Kekristenan. Kekristenan tidak ada hubungannya dengan Kristus. Sesungguhnya, kekristenan adalah anti-Kristus -- sama seperti Buddhisme adalah anti-Buddha dan Jainisme anti-Mahavir. Kristus memiliki sesuatu dalam dirinya yang tidak dapat diatur: sifat alami itu adalah pemberontakan dan pemberontakan tidak dapat diatur. Saat engkau mengaturnya, engkau membunuhnya. Kemudian mayat mati tetap ada. Engkau dapat menyembahnya, tapi engkau tidak dapat ditransformasikan oleh itu. Engkau dapat memikul beban selama berabad-abad dan berabad-abad, tapi itu hanya akan membebani dirimu, itu tidak akan membebaskanmu.

\english
I WILL SPEAK ON CHRIST, but not on Christianity. Christianity has nothing to do with Christ. In fact, Christianity is anti-Christ -- just as Buddhism is anti-Buddha and Jainism anti-Mahavir. Christ has something in him which cannot be organized: the very nature of it is rebellion and a rebellion cannot be organized. The moment you organize it, you kill it. Then the dead corpse remains. You can worship it, but you cannot be transformed by it. You can carry the load for centuries and centuries, but it will only burden you, it will not liberate you.

\bahasa
Itulah sebabnya, sejak awal, biarlah menjadi jelas: Aku mendukung semua untuk Kristus, tapi bahkan tidak sebagian kecilpun dariku  adalah untuk kekristenan. Jika engkau menginginkan Kristus, engkau harus melampaui kekristenan. Jika engkau melekat dengan kekristenan, engkau tidak akan dapat mengerti Kristus. Kristus berada di luar semua gereja.

\english
That's why, from the beginning, let it be absolutely clear: I am all for Christ, but not even a small part of me is for Christianity. If you want Christ, you have to go beyond Christianity. If you cling too much to Christianity, you will not be able to understand Christ. Christ is beyond all churches.

\bahasa
Kristus adalah prinsip dari agama. Di dalam Kristus semua aspirasi umat manusia terpenuhi. Dia adalah sintesis langka. Biasanya manusia hidup dalam penderitaan, kesedihan, kecemasan, kesakitan dan kesengsaraan. Jika engkau melihat Krishna, dia telah beralih ke polaritas lain: dia hidup dalam ekstase. Tidak ada penderitaan yang tersisa; kesedihan telah hilang. Engkau dapat mencintainya, engkau dapat menari dengannya untuk sementara waktu, tapi jembatannya akan hilang. Engkau dalam penderitaan, dia dalam ekstase - di mana jembatannya?

\english
Christ is the very principle of religion. In Christ all the aspirations of humanity are fulfilled. He is a rare synthesis. Ordinarily a human being lives in agony, anguish, anxiety, pain and misery. If you look at Krishna, he has moved to the other polarity: he lives in ecstasy. There is no agony left; the anguish has disappeared. You can love him, you can dance with him for a while, but the bridge will be missing. You are in agony, he is in ecstasy -- where is the bridge?

\bahasa
Seorang Buddha bahkan telah melangkah lebih jauh lagi. Dia tidak dalam penderitaan, juga tidak dalam ekstase. Dia benar-benar hening dan tenang. Dia begitu jauh sehingga engkau dapat melihatnya, tapi engkau tidak dapat mempercayai dia ada. Ia seperti mitos - mungkin pemenuhan keinginan umat manusia. Bagaimana mungkin orang seperti itu berjalan di bumi ini, sangat transenden terhadap semua penderitaan dan ekstase? Dia terlalu jauh.

\english
A Buddha has gone even farther away. He is neither in agony, nor in ecstasy. He is absolutely quiet and calm. He is so far away that you can look at him, but you cannot believe that he is. It looks like a myth -- maybe a wish fulfillment of humanity. How can such a man walk on this earth, so transcendental to all agony and ecstasy? He is too far away.

\bahasa
Yesus adalah puncak dari semua aspirasi. Dia sangat menderita seperti dirimu, setiap setiap manusia yang dilahirkan - dalam penderitaan di kayu salib. Dia berada dalam ekstase yang kadang-kadang Krishna capai: dia merayakannya; Dia adalah sebuah lagu, sebuah tarian. Dan dia juga transendensi. Ada saat-saat, ketika engkau mendekat dan mendekatinya, ketika engkau akan melihat bahwa keberadaan terdalamnya bukanlah salib atau perayaannya, tapi transendensi.

\english
Jesus is the culmination of all aspiration. He is in agony as you are, as every human being is born -- in agony on the cross. He is in the ecstasy that sometimes a Krishna achieves: he celebrates; he is a song, a dance. And he is also transcendence. There are moments, when you come closer and closer to him, when you will see that his innermost being is neither the cross nor his celebration, but transcendence.

\bahasa
Itulah keindahan Kristus: ada jembatan. Engkau dapat bergerak ke arahnya perlahan-lahan, dan dia dapat membawamu menuju yang tidak diketahui -- dan begitu perlahan sehingga engkau bahkan tidak akan sadar saat melintasi pembatas, saat engkau memasuki yang tidak diketahui dari yang diketahui, ketika dunia lenyap dan Tuhan muncul. Engkau dapat mempercayainya, karena dia sangat mirip denganmu namun begitu berbeda. Engkau dapat percaya padanya karena dia adalah bagian dari penderitaanmu; engkau dapat mengerti bahasanya.

\english
That's the beauty of Christ: there exists a bridge. You can move towards him by and by, and he can lead you towards the unknown -- and so slowly that you will not even be aware when you cross the boundary, when you enter the unknown from the known, when the world disappears and God appears. You can trust him, because he is so like you and yet so unlike. You can believe in him because he is part of your agony; you can understand his language.

\bahasa
Itulah sebabnya mengapa Yesus menjadi tonggak sejarah yang hebat dalam sejarah kesadaran. Bukan hanya kebetulan bahwa kelahiran Yesus telah menjadi tanggal paling penting dalam sejarah. Memang harus demikian. Sebelum Kristus, sebuah dunia; setelah Kristus, dunia yang sama sekali berbeda telah ada -- sebuah pembatas dalam kesadaran manusia. Ada begitu banyak kalender, begitu banyak cara, namun kalender yang didasarkan pada Kristus adalah yang paling signifikan. Bersama dia ada sesuatu yang berubah dalam diri manusia; bersama dia sesuatu telah merasuk ke dalam kesadaran manusia. Buddha itu indah, luar biasa, tapi bukan dari dunia ini; Krishna mudah tuk dicintai -- tapi tetap saja jembatannya hilang. Kristus adalah jembatan itu.

\english
That's why Jesus became a great milestone in the history of consciousness. It is not just coincidental that Jesus' birth has become the most important date in history. It has to be so. Before Christ, one world; after Christ, a totally different world has existed -- a demarcation in the consciousness of man. There are so many calendars, so many ways, but the calendar that is based on Christ is the most significant. With him something has changed in man; with him something has penetrated into the consciousness of man. Buddha is beautiful, superb, but not of this world; Krishna is lovable -- but still the bridge is missing. Christ is the bridge.

\bahasa
Oleh karena itu aku telah memilih untuk berbicara tentang Kristus. Tapi ingat selalu, aku tidak berbicara tentang kekristenan. Gereja selalu anti-Kristus. Begitu engkau mencoba mengorganisasi pemberontakan, pemberontakan harus mereda. Engkau tidak dapat mengorganisasi badai -- bagaimana engkau dapat mengorganisasi pemberontakan? Pemberontakan benar dan hidup hanya jika itu adalah kekacauan.

\english
Hence I have chosen to talk on Christ. But remember always, I am not talking on Christianity. The Church is always anti-Christ. Once you try to organize a rebellion, the rebellion has to be subsided. You cannot organize a storm -- how can you organize a rebellion? A rebellion is true and alive only when it is a chaos.

\bahasa
Bersama Yesus, sebuah kekacauan memasuki kesadaran manusia. Sekarang organisasi tidak harus dilakukan di luar, di masyarakat; keteraturan itu harus dibawa ke inti terdalam dari keberadaanmu. Kristus telah membawa kekacauan. Sekarang, karena kekacauan itu, engkau harus dilahirkan benar-benar baru, sebuah keteraturan yang berasal dari keberadaan terdalam -- bukan Gereja baru melainkan manusia baru, bukan masyarakat baru melainkan kesadaran manusia yang baru.

\english
With Jesus, a chaos entered into human consciousness. Now the organization is not to be done on the outside, in the society; the order has to be brought into the innermost core of your being. Christ has brought a chaos. Now, out of that chaos, you have to be born totally new, an order coming from the innermost being -- not a new Church but a new man, not a new society but a new human consciousness.

\bahasa
Itulah pesannya.

\english
That is the message.

\bahasa
Dan kata-kata ini dari Injil St. Yohanes -- engkau pasti pernah mendengarnya berkali-kali, engkau pasti pernah membacanya berkali-kali. Kata-kata ini telah menjadi hampir tidak berguna, sia-sia, tidak berarti, sepele. Kata-kata ini telah diulang berkali-kali sehingga sekarang tidak ada bel berbunyi di dalam dirimu saat engkau mendengarnya. Tapi kata-kata ini sangat potensial. Engkau mungkin telah kehilangan maknanya, tapi jika engkau menjadi sedikit waspada, sadar, arti dari kata-kata ini dapat direbut kembali. Itu akan menjadi perjuangan untuk merebut kembali maknanya ... sama seperti engkau merebut kembali sebuah daratan dari samudra.

\english
And these words from the gospel of St. John -- you must have heard them so many times, you must have read them so many times. They have become almost useless, meaningless, insignificant, trivial. They have been repeated so many times that now no bell rings within you when you hear them. But these words are tremendously potential. You may have lost the significance of them, but if you become a little alert, aware, the meaning of these words can be reclaimed. It is going to be a struggle to reclaim the meaning... just like you reclaim a land from the ocean.

\bahasa
Kekristenan telah membungkus kata-kata indah ini dengan begitu banyak interpretasi sehingga kesegaran aslinya hilang -- melalui mulut para pendeta yang hanya mengulangi seperti burung beo tanpa mengetahui apa yang mereka katakan: tanpa mengetahui, tanpa ragu sedikitpun, tanpa gemetaran dihadapan kesucian kata-kata ini. Mereka hanya mengulangi kata-kata seperti robot mekanis. Gerak-isyarat mereka salah, karena semuanya telah dilatih.

\english
Christianity has covered these beautiful words with so many interpretations that the original freshness is lost -- through the mouths of the priests who are simply repeating like parrots without knowing what they are saying: without knowing, without hesitating, without trembling before the sacredness of these words. They are simply repeating words like mechanical robots. Their gestures are false, because everything has been trained.

\bahasa
Pernah aku diundang ke sebuah perguruan teologi Kristen. Aku terkejut saat mereka membawaku berkeliling perguruan tinggi. Itu adalah salah satu perguruan tinggi teologi terbesar di India: setiap tahun mereka mempersiapkan dua ratus sampai tiga ratus pendeta Kristen dan misionaris di sana -- sebuah pelatihan lima tahun. Dan semuanya harus diajarkan: bagaimana cara berdiri di atas mimbar, bagaimana berbicara, dimana harus memberi penekanan lebih banyak, bagaimana menggerakkan tanganmu -- semuanya harus diajarkan. Kemudian semuanya menjadi salah, maka orang tersebut hanya memberi gerak-isyarat kosong.

\english
Once I was invited to a Christian theological college. I was surprised when they took me around the college. It is one of the greatest theological colleges in India: every year they prepare two hundred to three hundred Christian priests and missionaries there -- a five-year training. And everything has to be taught: even how to stand on the pulpit, how to speak, where to give more emphasis, how to move your hands -- everything has to be taught. Then everything becomes false, then the person is just making empty gestures.

\bahasa
Kata-kata ini seperti api, tapi selama berabad-abad pengulangan, pengulangan seperti burung beo, banyak debu yang berkerumun di sekitar api. Usahaku adalah untuk mengungkap kata-kata ini lagi. Waspadalah karena kita akan menginjak jalur yang terkenal dengan cara yang sangat tidak biasa, menginjak wilayah yang sangat terkenal dengan sikap yang sangat berbeda dan benar-benar baru. Wilayah ini akan menjadi tua. Usahaku akan memberimu kesadaran baru untuk melihatnya. Aku ingin meminjamkan mataku agar engkau dapat melihat hal-hal lama dalam cahaya baru. Dan saat engkau memiliki mata baru, semuanya menjadi baru. Dengarkan:

\english
These words are like fire, but through centuries of repetition, parrot-like repetition, much dust has gathered around the fire My effort will be to uncover them again. Be very alert because we will be treading on a well-known path in a very unknown way, treading on very well-known territory with a very different, totally new attitude. The territory is going to be old. My effort will be to give you a new consciousness to see it. I would like to lend you my eyes so that you can see the old things in new light. And when you have new eyes, everything becomes new. Listen:

\bahasa
PADA MULANYA ADALAH FIRMAN, DAN FIRMAN ITU ADA BERSAMA TUHAN, DAN FIRMAN ITU ADALAH TUHAN.

\english
In The Beginning Was The Word, And The Word Was With God, And The Word Was God.
